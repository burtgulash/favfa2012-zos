\documentclass[titlepage]{article}
\usepackage[utf8]{inputenc}
\usepackage[czech]{babel}


\begin{document}
\begin{titlepage}
\begin{center}
	\mbox{} \\[3cm]
	\huge{Semestrální práce z předmětu KIV/ZOS} \\[2.5cm]
	\Large{Tomáš Maršálek, A10B0632P} \\
	\large{marsalet@students.zcu.cz} \\[1cm]
	\normalsize{\today}
\end{center}
\thispagestyle{empty}
\end{titlepage}

\section{Problém spícího holiče}
Ve městě sídlí jeden holič, který má pouze jedno křeslo, ve kterém holí
zákazníky. Když do jeho holičství přijde zákazník, usadí se v čekárně a bude
čekat, dokud ho holič nepřijme do křesla. Čekárna má ovšem omezený počet míst,
tedy v případě plné čekárny každý další zákazník musí odejít.

\section{Řešení}
Program je v jazyce C, pro paralelizaci jsou použita vlákna knihovny
\textbf{pthread}.

\subsection{Správa vláken}
Pro vytvoření vlákna je použita funkce \textbf{pthread\_create}, u které je
třeba kontrolovat vrácenou hodnotu. V případě hodnoty různé od nuly se
nepodařilo vytvořit vlákno a program je raději ukončen.  

Na konci programu se čeká na všechna vytvořená vlákna pomocí
\textbf{pthread\_join}. Bez použití tohoto čekání by program skončil dříve, než
by všechna vlákna dokončila svou práci.

\subsection{Synchronizace}
Celkem je použito sedm semaforů. Čtyři pro samotný korektní běh problému
spícího holiče a dva semafory čistě kvůli synchronizaci výpisů holiče a
zákazníka. 

\begin{enumerate}
	\item {\em customer\_ready} - zákazníci dávají vědět holiči, že jsou
připraveni k ostříhání.
	\item {\em barber\_ready} - holič dává vědět zákazníkům, že je volné křeslo.
	\item {\em queue\_empty} - celočíselná hodnota, udává počet volných míst v
čekárně. Je chráněna binárním semaforem {\em queue\_lock}.
	\item {\em n\_lock} - chrání proměnnou {\em n}, aby nedošlo k dvojímu
zápisu. Proměnná {\em n} udává počet zákazníků, kteří buď skončili v čekárně
nebo odešli kvůli plné čekárně.
	\item {\em print\_lock} - zámek kolem každého výpisu, aby náhodou nedošlo k
vypisování ze dvou vláken najednou.
	\item {\em becut\_sync} - čistě synchronizační semafor pro vlákna
zákazníků, aby holičův výpis proběhl před zákazníkovým.
	\item {\em cut\_sync} - stejná funkce, jen pro vlákno holiče.
\end{enumerate}

\section{Použití programu}
Program spustíme z konzole buďto bez parametrů, kde počet zákazníků bude na
výchozí hodnotě 17 a velikost čekárny 4.
\begin{verbatim}
$ ./sleepingbarber
\end{verbatim}

Chceme-li parametry modifikovat, zadáme nejprve počet zákazníků, poté velikost
čekárny.
\begin{verbatim}
$ ./sleepingbarber 8 4
\end{verbatim}


\section{Ukázka běhu programu}
\begin{verbatim}
Number of customers: 8
Queue size         : 4

START
Customer 0 waits.
Barber cuts hair.
Customer 0 is cut.
Customer 1 waits.
Barber cuts hair.
Customer 1 is cut.
Customer 3 waits.
Barber cuts hair.
Customer 3 is cut.
Customer 2 waits.
Barber cuts hair.
Customer 2 is cut.
Customer 4 waits.
Barber cuts hair.
Customer 4 is cut.
Customer 5 waits.
Barber cuts hair.
Customer 5 is cut.
Customer 6 waits.
Customer 7 waits.
Barber cuts hair.
Customer 6 is cut.
Barber cuts hair.
Customer 7 is cut.
END
\end{verbatim}
\end{document}
